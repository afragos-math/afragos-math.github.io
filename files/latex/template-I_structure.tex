\documentclass[twoside, A4]{book}

%Πακέτα.
	\usepackage[greek,english]{babel}
	\usepackage[utf8]{inputenc}
	\usepackage{alphabeta}
	\usepackage{array}
	\usepackage{amsfonts}
	\usepackage{amsmath}
	\usepackage{amssymb}
	\usepackage{amsthm}
	\usepackage{float}
	\usepackage{yhmath}
	\usepackage{color}
	\usepackage{graphicx}
	\usepackage{relsize}
	\usepackage{geometry}
	\usepackage{hyperref}
	\usepackage{titlesec}
	\usepackage{pgf,tikz,pgfplots}
	\usepackage{titletoc}
	\usepackage{lettrine}
	\usepackage{subcaption}
	\usepackage{fancyhdr}
	\usepackage{paracol}
	\usepackage{accents}
	\usepackage{manfnt}
	\usepackage{multicol}
	\usepackage{imakeidx}
	\usepackage{kerkis}
	\usepackage{afterpage}
	\usepackage{paracol}
	\usepackage{calrsfs} \DeclareMathAlphabet{\pazocal}{OMS}{zplm}{m}{n}

%Τέλος Πακέτα.
\graphicspath{{pictures/}} %Όλες οι φωτογραφίες/εικόνες διαβάζονται από τον φάκελο «pictures».

\geometry{
	paper=a4paper,
	top=2cm,
	bottom=2cm,
	left=3.3cm,
	right=2cm,
	headheight=12pt,
	footskip=1.2cm,
	headsep=11pt,
}

\hypersetup{
    colorlinks=true, 
    linktoc=all,
    linkcolor=blue,      
}

\titleformat{\chapter}
[display]
{\bfseries\Huge}
{{\Large \bfseries\MakeUppercase {ΚΕΦΑΛΑΙΟ}} {\huge \bfseries \thechapter}}
{1pt}
{\flushright \titlerule[3pt]\vspace{12pt}}
{\titlespacing{\chapter}{0pt}{0pt}{0pt}}

\makeindex

\sloppy
\columnratio{0.15}

\addto\captionsenglish{\renewcommand*\contentsname{\centering \textbf{Πίνακας Περιεχομένων}}}

\AtBeginDocument{\renewcommand\bibname{Βιβλιογραφία}}

\pagestyle{fancy}

\renewcommand{\chaptermark}[1]{\markboth{\normalsize \emph{\textbf{Κεφάλαιο \thechapter.\ }} #1}{}}
\renewcommand{\sectionmark}[1]{\markright{\normalsize\thesection\hspace{5pt}#1}{}} 

\fancyhf{}
\fancyhead[LE,RO]{\normalsize\thepage} 
\fancyhead[LO]{\rightmark} 
\fancyhead[RE]{\leftmark}


\renewcommand{\headrulewidth}{0.5pt}

\fancypagestyle{plain}{
	\fancyhead{}\renewcommand{\headrulewidth}{0pt}
}

\makeatletter
\renewcommand{\cleardoublepage}{
\clearpage\ifodd\c@page\else
\hbox{}
\vspace*{\fill}
\thispagestyle{empty}
\newpage
\fi}

%Κουτάκια και enviroments.
\newtheorem{theoremTemp}{Θεώρημα}[section]

\newtheoremstyle{theoremTemp}
{0pt}
{0pt}
{}
{}
{\bf}
{\;}
{0.25em}
{\thmnumber{{#1}{}~{#2}}
\thmnote{~(#3).}}

\newenvironment{theorem}{~\\ \begin{tabular}{!{\vrule width 3pt} p{.965\textwidth} |}\hline\begin{theoremTemp}}{\end{theoremTemp}\\ \hline\end{tabular}\\ \par}

\newtheorem{propositionTemp}{Πρόταση}[section]

\newtheoremstyle{propositionTemp}
{0pt}
{0pt}
{\normalfont}
{}
{\bf}
{\;}
{0.25em}
{\thmnumber{{#1}{}~{#2}}
\thmnote{~(#3).}}

\newenvironment{proposition}{~\\ \begin{tabular}{!{\vrule width 3pt} p{.965\textwidth} |}\hline\begin{propositionTemp}}{\end{propositionTemp}\\ \hline\end{tabular}\\ \par}

\newtheorem{definitionTemp}{Ορισμός}[section]

\newtheoremstyle{propositionTemp}
{0pt}
{0pt}
{\normalfont}
{}
{\bf}
{\;}
{0.25em}
{\thmnumber{{#1}{}~{#2}}
\thmnote{~(#3).}}

\newenvironment{definition}{~\\ \begin{tabular}{!{\vrule width 3pt} p{.965\textwidth} |}\hline\begin{definitionTemp}}{\end{definitionTemp}\\ \hline\end{tabular}\\ \par}

\newtheorem*{nonameTemp}{..}

\newtheoremstyle{nonameTemp}
{0pt}
{0pt}
{}
{}
{\bf}
{\;}
{0.25em}
{}

\newenvironment{noname}{~\\ \begin{tabular}{!{\vrule width 3pt} p{.965\textwidth} |}\hline\begin{nonameTemp}}{\end{nonameTemp}\\ \hline\end{tabular}\\ \par}

\newtheorem{lemma}{Λήμμα}[section]

\newtheoremstyle{Λήμμα}
{0pt}
{0pt}
{}
{}
{\bf}
{\;}
{0.25em}
{\thmnumber{{#1}{}~{#2}}
\thmnote{~(#3).}}

\newtheorem{corollary}{Πόρισμα}[section]

\newtheoremstyle{corollary}
{0pt}
{0pt}
{}
{}
{\bf}
{\;}
{0.25em}
{\thmnumber{{#1}{}~{#2}}
\thmnote{~(#3).}}

\newtheorem{remark}{Παρατήρηση}[section]

\newtheoremstyle{remark}
{0pt}
{0pt}
{}
{}
{\bf}
{\;}
{0.25em}
{\thmnumber{{#1}{}~{#2}}
\thmnote{~(#3).}}

\newtheorem{exercise}{Άσκηση}[chapter]

\newtheoremstyle{exercise}
{0pt}
{0pt}
{}
{}
{\bf}
{\;}
{0.25em}
{\thmnumber{{#1}{}~{#2}}
\thmnote{~(#3).}}

\AtBeginDocument{\renewcommand{\arraystretch}{1.3}}
%Τέλος κουτάκια και enviroments.