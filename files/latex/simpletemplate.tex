\documentclass[A4]{article}

%Πακέτα, χρήσιμα στη LaTeX

%Κάπως κύρια
\usepackage[utf8]{inputenc}
\usepackage[greek,english]{babel}
\usepackage{alphabeta}
\usepackage{amsmath}
\usepackage{amssymb}

%Κάπως δευτερεύοντα
\usepackage{hyperref}
\usepackage{graphicx}
\usepackage{tikz}

%(Προαιρετικό) Φτιάχνουμε τα links, ώστε να είναι πιο όμορφα.
\hypersetup{
    colorlinks=true, 
    linktoc=all,
    linkcolor=blue,      
}

\begin{document}
%Κατασκευή του τίτλου
	\title{\textbf{Απλό template στη \LaTeX}}
	\author{Κουκουδάκης Ν.\footnote{\href{mailto://nicolaskoukoudakis@gmail.com}{nicolaskoukoudakis@gmail.com}}, Φράγκος Α\footnote{\href{mailto://afragos@math.uoa.gr}{afragos@math.uoa.gr}}.}
	\date{Πέμπτη 14 Δεκεμβρίου 2023}
	\maketitle
	
%Περιεχόμενα (Φτιάχνονται αυτόματα)
	%Αλλαγή του ονόματος των περιεχομένων, από Contents σε Περιεχόμενα.
	\renewcommand*\contentsname{\textbf{Περιεχόμενα}}
	\tableofcontents

\section{Παράγραφος}
\subsection{Υποπαράγραφος}
\end{document}